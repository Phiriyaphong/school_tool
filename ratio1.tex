% Options for packages loaded elsewhere
% Options for packages loaded elsewhere
\PassOptionsToPackage{unicode}{hyperref}
\PassOptionsToPackage{hyphens}{url}
\PassOptionsToPackage{dvipsnames,svgnames,x11names}{xcolor}
%
\documentclass[
  a4paper,
  DIV=11,
  numbers=noendperiod]{scrartcl}
\usepackage{xcolor}
\usepackage[top= 1.0in,left= 1.0in,right = 1.0in,bottom =
1.0in,heightrounded]{geometry}
\usepackage{amsmath,amssymb}
\setcounter{secnumdepth}{-\maxdimen} % remove section numbering
\usepackage{iftex}
\ifPDFTeX
  \usepackage[T1]{fontenc}
  \usepackage[utf8]{inputenc}
  \usepackage{textcomp} % provide euro and other symbols
\else % if luatex or xetex
  \usepackage{unicode-math} % this also loads fontspec
  \defaultfontfeatures{Scale=MatchLowercase}
  \defaultfontfeatures[\rmfamily]{Ligatures=TeX,Scale=1}
\fi
\usepackage{lmodern}
\ifPDFTeX\else
  % xetex/luatex font selection
\fi
% Use upquote if available, for straight quotes in verbatim environments
\IfFileExists{upquote.sty}{\usepackage{upquote}}{}
\IfFileExists{microtype.sty}{% use microtype if available
  \usepackage[]{microtype}
  \UseMicrotypeSet[protrusion]{basicmath} % disable protrusion for tt fonts
}{}
\usepackage{setspace}
\makeatletter
\@ifundefined{KOMAClassName}{% if non-KOMA class
  \IfFileExists{parskip.sty}{%
    \usepackage{parskip}
  }{% else
    \setlength{\parindent}{0pt}
    \setlength{\parskip}{6pt plus 2pt minus 1pt}}
}{% if KOMA class
  \KOMAoptions{parskip=half}}
\makeatother
% Make \paragraph and \subparagraph free-standing
\makeatletter
\ifx\paragraph\undefined\else
  \let\oldparagraph\paragraph
  \renewcommand{\paragraph}{
    \@ifstar
      \xxxParagraphStar
      \xxxParagraphNoStar
  }
  \newcommand{\xxxParagraphStar}[1]{\oldparagraph*{#1}\mbox{}}
  \newcommand{\xxxParagraphNoStar}[1]{\oldparagraph{#1}\mbox{}}
\fi
\ifx\subparagraph\undefined\else
  \let\oldsubparagraph\subparagraph
  \renewcommand{\subparagraph}{
    \@ifstar
      \xxxSubParagraphStar
      \xxxSubParagraphNoStar
  }
  \newcommand{\xxxSubParagraphStar}[1]{\oldsubparagraph*{#1}\mbox{}}
  \newcommand{\xxxSubParagraphNoStar}[1]{\oldsubparagraph{#1}\mbox{}}
\fi
\makeatother


\usepackage{longtable,booktabs,array}
\usepackage{calc} % for calculating minipage widths
% Correct order of tables after \paragraph or \subparagraph
\usepackage{etoolbox}
\makeatletter
\patchcmd\longtable{\par}{\if@noskipsec\mbox{}\fi\par}{}{}
\makeatother
% Allow footnotes in longtable head/foot
\IfFileExists{footnotehyper.sty}{\usepackage{footnotehyper}}{\usepackage{footnote}}
\makesavenoteenv{longtable}
\usepackage{graphicx}
\makeatletter
\newsavebox\pandoc@box
\newcommand*\pandocbounded[1]{% scales image to fit in text height/width
  \sbox\pandoc@box{#1}%
  \Gscale@div\@tempa{\textheight}{\dimexpr\ht\pandoc@box+\dp\pandoc@box\relax}%
  \Gscale@div\@tempb{\linewidth}{\wd\pandoc@box}%
  \ifdim\@tempb\p@<\@tempa\p@\let\@tempa\@tempb\fi% select the smaller of both
  \ifdim\@tempa\p@<\p@\scalebox{\@tempa}{\usebox\pandoc@box}%
  \else\usebox{\pandoc@box}%
  \fi%
}
% Set default figure placement to htbp
\def\fps@figure{htbp}
\makeatother





\setlength{\emergencystretch}{3em} % prevent overfull lines

\providecommand{\tightlist}{%
  \setlength{\itemsep}{0pt}\setlength{\parskip}{0pt}}



 


%%Thai-preamble.tex

%% ---- Preamble Injected from thaipdf package (BEGIN) ---- %%

%% ขอบคุณแหล่งข้อมูลต่อไปนี้ สำหรับคำแนะนำการตั้งค่าภาษาไทยใน LaTeX %%
%   ฑิตยา หวานวารี: http://pioneer.netserv.chula.ac.th/~wdittaya/LaTeX/LaTeXThai.pdf
%   mathmd from Github; https://github.com/mathmd/polygloTeX

%% ---------------- เริ่มการตั้งค่าภาษาไทย ---------------- %%

%% ---- ตั้งค่าให้ตัดคำภาษาไทย ---- %%
\XeTeXlinebreaklocale "th"
\XeTeXlinebreakskip = 0pt plus 0pt % เพิ่มความกว้างเว้นวรรคให้ความยาวแต่ละบรรทัดเท่ากัน

%% ---- font settings ---- %%
\usepackage{fontspec} % For Thai font
\defaultfontfeatures{Mapping=tex-text} % map LaTeX formating, e.g., ``'', to match the current font

% To change the main font, uncomment one of the below command, and make sure you have these fonts installed.
% \setmainfont{TeX Gyre Termes} % Free Times
% \setsansfont{TeX Gyre Heros} % Free Helvetica
% \setmonofont{TeX Gyre Cursor} % Free Courier

% ตั้งฟอนต์หลักภาษาไทย ที่น่าใช้ เช่น: "TH Sarabun New", "Laksaman"
\newfontfamily{\thaifont}[Scale=MatchUppercase,Mapping=textext]{TH
Sarabun New}
\newenvironment{thailang}{\thaifont}{} % create environment for Thai language
\usepackage[Latin,Thai]{ucharclasses} % ตั้งค่าให้ใช้ "thailang" environment เฉพาะ string ที่เป็น Unicode ภาษาไทย

\setTransitionTo{Thai}{\begin{thailang}} % Use environment "thailang" when found Thai font.
\setTransitionFrom{Thai}{\end{thailang}} % If not found, end the environment.
\renewcommand{\figurename}{รูปที่}
%% ---- Spacing ---- %%
\renewcommand{\baselinestretch}{1.5} % Line Spacing (1.5 is recommended for Thai language)


%% ---- using alphabatic language ---- %%
\usepackage{polyglossia}
\setdefaultlanguage{english} % it is preferrable to set English as the main language, since the numeric system is compatible with most LaTeX features such as 'enumerate' and so on
\setotherlanguages{thai}

\AtBeginDocument\captionsthai % allow captions to be in Thai

%% ---------------- จบการตั้งค่าภาษาไทย! ---------------- %%
%% ---------------- สามารถใส่การตั้งค่าอื่นๆ เพิ่มเติมได้ ---------------- %%
\usepackage{graphicx}
\usepackage{tabto}
\usepackage{fancyhdr}
\pagestyle{fancy}
\usepackage{multicol}
\usepackage{titling}
\usepackage{amsfonts}

\usepackage{tcolorbox}

\usepackage{multirow}
     \usepackage{sectsty}
     \sectionfont{\centering}
\renewcommand{\figurename}{รูปที่}

\setlength\columnwidth{1cm}

\setlength\columnsep{0.5cm}

%% ---- hyperref settings (uncomment for colorful link) ---- %%
% \usepackage{hyperref}
% \usepackage{url}
% \usepackage{cite}
% \usepackage{xcolor}
% \hypersetup{
%     colorlinks, % for colorful link
%     linkcolor={red!50!black},
%     citecolor={blue!50!black},
%     urlcolor={blue!80!black}
%     }

%% ---- Preamble Injected from thaipdf package (END) ---- %%


\KOMAoption{captions}{tableheading}
\makeatletter
\@ifpackageloaded{caption}{}{\usepackage{caption}}
\AtBeginDocument{%
\ifdefined\contentsname
  \renewcommand*\contentsname{Table of contents}
\else
  \newcommand\contentsname{Table of contents}
\fi
\ifdefined\listfigurename
  \renewcommand*\listfigurename{List of Figures}
\else
  \newcommand\listfigurename{List of Figures}
\fi
\ifdefined\listtablename
  \renewcommand*\listtablename{List of Tables}
\else
  \newcommand\listtablename{List of Tables}
\fi
\ifdefined\figurename
  \renewcommand*\figurename{Figure}
\else
  \newcommand\figurename{Figure}
\fi
\ifdefined\tablename
  \renewcommand*\tablename{Table}
\else
  \newcommand\tablename{Table}
\fi
}
\@ifpackageloaded{float}{}{\usepackage{float}}
\floatstyle{ruled}
\@ifundefined{c@chapter}{\newfloat{codelisting}{h}{lop}}{\newfloat{codelisting}{h}{lop}[chapter]}
\floatname{codelisting}{Listing}
\newcommand*\listoflistings{\listof{codelisting}{List of Listings}}
\makeatother
\makeatletter
\makeatother
\makeatletter
\@ifpackageloaded{caption}{}{\usepackage{caption}}
\@ifpackageloaded{subcaption}{}{\usepackage{subcaption}}
\makeatother
\usepackage{bookmark}
\IfFileExists{xurl.sty}{\usepackage{xurl}}{} % add URL line breaks if available
\urlstyle{same}
\hypersetup{
  pdfauthor={พิริยพงศ์ หริ่งรอด},
  colorlinks=true,
  linkcolor={blue},
  filecolor={Maroon},
  citecolor={Blue},
  urlcolor={Blue},
  pdfcreator={LaTeX via pandoc}}


\author{พิริยพงศ์ หริ่งรอด}
\date{26 May 2025}
\begin{document}


\setstretch{1}
\lhead{}
\chead{- \thepage \ -}
\rhead{ม.2 อัตราส่วน สัดส่วน ร้อยละ}
\renewcommand{\headrulewidth}{0.4pt}
\lfoot{\today}
\cfoot{}
\rfoot{Aj.Prem}
\renewcommand{\footrulewidth}{0.4pt}
\newtcolorbox{mybox}{colback=white!5!white,
colframe=red!30!black}

\definecolor{customblue}{HTML}{07C2F6}

\section{อัตราส่วน สัดส่วน
และร้อยละ}\label{uxe2duxe15uxe23uxe32uxe2auxe27uxe19-uxe2auxe14uxe2auxe27uxe19-uxe41uxe25uxe30uxe23uxe2duxe22uxe25uxe30}

\subsection{อัตราส่วน}\label{uxe2duxe15uxe23uxe32uxe2auxe27uxe19}

นักเรียนอาจพบข้อความที่แสดงความสัมพันธ์ของปริมาณสองปริมาณในชีวิตประจำวัน เช่น

\centering

``ถ้าหุงข้าวกล้องให้อร่อย หากมีข้าวกล้องจำนวน \(1\) ถ้วย จะต้องใช้น้ำจำนวน \(2\)
ถ้วย''

\flushleft

ซึ่งเป็นข้อความที่แสดงความสัมพันธ์ของ \makebox[3cm]{\dotfill} กับ ปริมาณของ
\makebox[3cm]{\dotfill}

หรือในตลาดบางครั้งนักเรียนอาจได้ยินแม่ค้าร้องขายของว่่า

\centering

``ผักทุกอย่าง \(3\) กำ \(20\) บาท''

\flushleft

ซึ่งเป็นข้อความที่แสดงความสัมพันธ์ของ \makebox[3cm]{\dotfill} กับ ปริมาณของ
\makebox[3cm]{\dotfill}

\begin{mybox}

ความสัมพันธ์ที่แสดงการเปรียบเทียบปริมาณสองปริมาณ ซึ่งอาจมีหน่วยเดียวกันหรือหน่วยต่างกันก็ได้ เรียกว่า \textbf{\textcolor{NavyBlue}{อัตราส่วน (ratio)}}

\end{mybox}

กล่าวคือ อัตราส่วนของปริมาณ \(a\) ต่อปริมาณ \(b\) เขียนแทนด้วย \(a:b\) อ่านว่า
\(a:b\)

\renewcommand{\arraystretch}{1.5}

\begin{tabular}{lll}
เรียก    & $a$ & ว่าจำนวนแรกหรือจำนวนที่หนึ่งของอัตราส่วน \\
และเรียก & $b$ &ว่าจำนวนหลังหรือจำนวนที่สองของอัตราส่วน \\
\end{tabular}

\vspace{-7pt}

โดยเราจะพิจารณาอัตราส่วน \(a\) ต่อ \(b\) ในกรณีที่ \(a\) และ \(b\)
เป็น\textbf{จำนวนบวก} เท่านั้น

\begin{tabular}{lp{12cm}}

จากข้อความ & ``ถ้าหุงข้าวกล้องให้อร่อย หากมีข้าวกล้องจำนวน $1$ ถ้วย จะต้องใช้น้ำจำนวน $2$ ถ้วย" \\

          & เขียนให้อยู่ในรูปอัตราส่วนของ\textbf{ปริมาณข้าวกล้อง} ต่อ \textbf{ปริมาณน้ำ}โดยปริมาตร \\
          & ได้เป็น \dotfill \\

และจากข้อความ  & ``ผักทุกอย่าง $3$ กำ $20$ บาท" \\

          & เขียนให้อยู่ในรูปอัตราส่วนของ\textbf{ปริมาณผักเป็นกำ} ต่อ\textbf{ราคาเป็นบาท}\\
          & ได้เป็น \dotfill \\

\end{tabular}

\begin{tcolorbox}[colback=white!5!white,
colframe=red!70!black]

อัตราส่วนที่แสดงการเปรียบเทียบปริมาณสองปริมาณที่มีหน่วยเดียวกัน และมีความชัดเจนว่าเป็นหน่วยของสิ่งใด เช่น น้ำหนักหรือจำนวนคน เราไม่นิยมเขียนหน่วยกำกับไว้
เช่น อัตราส่วนของน้ำหนักหญ้าสดต่อน้ำหนักมูลไก่เป็น $50:5$
\tcblower
อัตราส่วนที่แสดงการเปรียบเทียบปริมาณสองปริมาณที่มีหน่วยต่างกัน เราจะเขียนหน่วยกำกับไว้ที่คำอธิบาย เช่น อัตราส่วนของจำนวนไข่ไก่เป็น \textbf{ฟอง} ต่อราคาเป็น\textbf{บาท} เป็น $10:32$ หรือเขียนหน่วยกำกับไว้ในอัตราส่วน เช่น อัตราส่วนของจำนวนไข่ไก่ต่อราคาเป็น $10$ \textbf{ฟอง} $: 32$ \textbf{บาท}
\end{tcolorbox}

\newpage
\centering
\begin{tcolorbox}[colback=white!5!white,
colframe=red!70!black,width=5cm]
\centering \textbf{ลองทำดู}
\end{tcolorbox}

\flushleft

\begin{enumerate}
\def\labelenumi{\arabic{enumi}.}
\tightlist
\item
  จงเขียนอัตราส่วนจากข้อความต่อไปนี้
\end{enumerate}

\begin{tabular}[l]{p{0.5cm}lp{12cm}}

&1.1)  & ครู $2$ คน ดูแลนักเรียน $55$ คน \\
&      &  \dotfill \\

&1.2)  & นักเรียนใช้คอมพิวเตอร์เครื่องละ $3$ คน \\
&      &  \dotfill \\

&1.3)  & กรรไกร $3$ เล่ม สำหรับนักเรียน $10$ คน \\
&      &  \dotfill \\

&1.4)  & ราคาทองคำแท่งบาทละ $50,500$ บาท \\
&      &  \dotfill \\


&1.5)  & รถยนต์แล่นได้ระยะทาง $180$ กิโลเมตร ในเวลา $3$ ชั่วโมง \\
&      &  \dotfill \\


&1.6)  & อัตราการเต้นของของหัวใจมนุษย์เป็น $72$ ครั้งต่อนาที \\
&      &  \dotfill \\

&1.7)  & รูปสี่เหลี่ยมมุมฉาก มีความยาว $6$ เซนติเมตร และมีความกว้าง $4$ เซนติเมตร \\
&      &  \dotfill \\


&1.8)  & เตรียมน้ำเกลือสำหรับแช่ผลไม้โดยใช้เกลือ $2$ ช้อนชาและน้ำ $5$ ลิตร \\
&      &  \dotfill \\


&1.9)  & ชงกาแฟโดยใช้กาแฟ $2$ ช้อนชา และน้ำตาล $1$ ช้อนชา และน้ำตาล $1$ ช้อนชา \\
&      &  \dotfill \\


&1.10)  & แม่ซื้อมะม่วงและชมพู่อย่างละ $3$ กิโลกรัม \\
&      &  \dotfill \\

\end{tabular}

\begin{enumerate}
\def\labelenumi{\arabic{enumi}.}
\setcounter{enumi}{1}
\tightlist
\item
  ช่างปูกระเบื้องใช้กระเบื้องสีขาวและสีชมพูปูพื้นห้อง ดังรูป
\end{enumerate}

\begin{tabular}{p{0.5cm}clp{8cm}}
& 
\multirow{5}*{\includegraphics[width=5cm]{images/tile_ratio.png}} & \multicolumn{2}{l}{จงเขียนอัตราส่วนต่อไปนี้} \\

&   & 1)  & อัตราส่วนของจำนวนกระเบื้องสีชมพูต่อจำนวนกระเบื้องสีขาว \\
&   &     & \dotfill \\

&   & 2)  & อัตราส่วนของจำนวนกระเบื้องสีชมพูต่อจำนวนกระเบื้องทั้งหมด \\
&   &     & \dotfill \\

\end{tabular}

\newpage

\begin{enumerate}
\def\labelenumi{\arabic{enumi}.}
\setcounter{enumi}{2}
\tightlist
\item
  จากตารางแสดงจำนวนคนที่ชอบทานข้าวผัดและก๋วยเตี๋ยว (แต่ละคนชอบทานเพียงอย่างเดียว)
  ดังนี้
\end{enumerate}

\centering

\begin{tabular}[c]{l|cc|}
\cline{2-3}
                                 & \multicolumn{2}{c|}{\textbf{เพศ}} \\ \hline
\multicolumn{1}{|l|}{\textbf{รายการอาหาร}} & \multicolumn{1}{c|}{\textbf{ชาย (คน)}} & \textbf{หญิง (คน)} \\ \hline
\multicolumn{1}{|l|}{ข้าวผัด}    & \multicolumn{1}{c|}{100}   & 160  \\ \hline
\multicolumn{1}{|l|}{ก๋วยเตี๋ยว} & \multicolumn{1}{c|}{720}   & 645  \\ \hline
\end{tabular}

\flushleft

\begin{tabular}{p{0.5cm}lp{13cm}}

& \multicolumn{2}{l}{จงใช้ข้อมูลในตารางเขียนอัตราส่วนแสดงการเปรียเทียบจำนวนคนต่อไปนี้} \\

& 1) & จำนวนผู้ชายที่ชอบทานข้าวผัดต่อจำนวนผู้ชายที่ชอบทานก๋วยเตี๋ยว \\

&    &  \dotfill \\

& 2) & จำนวนผู้หญิงที่ชอบทานข้าวผัดต่อจำนวนหญิงที่ชอบทานก๋วยเตี๋ยว \\

&    &  \dotfill \\

& 3) & จำนวนผู้ชายที่ชอบทานก๋วยเตี๋ยวต่อจำนวนผู้หญิงที่ชอบทานก๋วยเตี๋ยว \\

&    &  \dotfill \\

& 4) & จำนวนผู้ชายต่อจำนวนผู้หญิง \\

&    &  \dotfill \\



\end{tabular}

\begin{enumerate}
\def\labelenumi{\arabic{enumi}.}
\setcounter{enumi}{3}
\tightlist
\item
  พลอยและแพรสั่งก๋วยเตี๋ยวน้ำมารับประทานคนละ \(1\) ชาม แต่ละคนปรุงรสดังนี้
\end{enumerate}

\begin{tabular}{p{0.5cm}llll}

& พลอย  & ใส่น้ำปลา $1$ ช้อนชา & น้ำมะนาว $3$ ช้อนชา & และน้ำตาลทราย $2$ ช้อนชา \\

& แพร  & ใส่น้ำปลา $2$ ช้อนชา & น้ำมะนาว $3$ ช้อนชา & และน้ำตาลทราย $2$ ช้อนชา \\

& \multicolumn{4}{l}{นักเรียนคิดว่าก๋วยเตี๋ยวของใครจะมีรสเค็มมากกว่า}\\

& \multicolumn{4}{p{10cm}}{\dotfill}\\



\end{tabular}

\newpage

\subsubsection{อัตราส่วนที่เท่ากัน}\label{uxe2duxe15uxe23uxe32uxe2auxe27uxe19uxe17uxe40uxe17uxe32uxe01uxe19}

พิจารณาข้อความต่อไปนี้

\centering

``ถ้าจะหุงข้าวกล้องให้อร่อย ข้าวกล้อง \(1\) ถ้วย จะต้องใช้น้ำ \(2\) ถ้วย''

\vspace{-10pt}

\flushleft

จากข้อความดังกล่าวเราสามารถเขียนอัตราส่วนของปริมาณข้าวกล้องต่อปริมาณน้ำได้เป็น
\dotfill

ถ้าต้องการหุงข้าวกล้องตามจำนวนที่กำหนดในตารางด้านล่าง นักเรียนคิดว่า จะต้องใช้น้ำกี่ถ้วย

จงเติมปริมาณน้ำที่ใช้ในการหุงข้าวกล้องในตารางให้สมบูรณ์

\centering

\begin{tabular}{c|p{1.5cm}|p{1.5cm}|p{1.5cm}|p{1.5cm}|p{1.5cm}}

\hline

ปริมาณข้าวกล้อง (ถ้วย) & \multicolumn{1}{c|}{$1$} 
                    & \multicolumn{1}{c|}{$2$}
                    & \multicolumn{1}{c|}{$3$}
                    & \multicolumn{1}{c|}{$4$}
                    & \multicolumn{1}{c}{$5$}\\


\hline

ปริมาณน้ำ (ถ้วย) &  
                    & 
                    & 
                    & 
                    & \\

\hline



\end{tabular}

เมื่อนักเรียนเติมตารางสมบูรณ์แล้ว สังเกตได้ว่าอัตราส่วนของปริมาณข้าวกล้องต่อปริมาณน้ำเป็น
ดังนี้

\centering

\makebox[1.5cm]{\dotfill} หรือ \makebox[1.5cm]{\dotfill} หรือ
\makebox[1.5cm]{\dotfill} หรือ \makebox[1.5cm]{\dotfill} หรือ
\makebox[1.5cm]{\dotfill}

\flushleft

ซึ่งอัตราส่วนเหล่านี้ ได้มาจากการหุงข้าวกล้องโดยใช้ปริมาณ \textbf{แบบเดียวกัน}
คือข้าวกล้อง \(1\) ถ้วย จะต้องใช้น้ำ \(2\) ถ้วย กล่าวคือ

\begin{tabular}{lcp{10cm}}
อัตราส่วน $2:4$  & สามารถเขียนได้เป็น & $1\times 2 : 2\times 2$ \\
อัตราส่วน $3:6$  & สามารถเขียนได้เป็น &  \dotfill                       \\
อัตราส่วน $4:8$  & สามารถเขียนได้เป็น &    \dotfill                      \\
อัตราส่วน $5:10$ & สามารถเขียนได้เป็น &   \dotfill                      \\
\end{tabular}

นอกจากนี้

\begin{tabular}{lcp{10cm}}

อัตราส่วน $1:2$   & สามารถเขียนได้เป็น & $2\div 2 : 4\div 2$ \\
                & หรือ &  \dotfill                       \\
                & หรือ &    \dotfill                      \\
                & หรือ &   \dotfill                      \\

\end{tabular}

พิจารณาการหาเศษส่วนที่เท่ากันโดยใช้วิธีการคูณหรือการหารด้วยจำนวนเดียวกันที่นักเรียนเคยทราบมาแล้ว

\centering

\renewcommand{\arraystretch}{2.5}

\begin{tabular}{c|c}

\hline
\textbf{คูณด้วยจำนวนเดียวกัน}  & \textbf{หารด้วยจำนวนเดียวกัน} \\

\hline
$\cfrac{1}{2}=\cfrac{1\times 2}{2\times 2}=\cfrac{2}{4}$ & 
$\cfrac{2}{4}=\cfrac{2\div 2}{4\div 2}=\cfrac{1}{2}$ \\

\hline
$\cfrac{1}{2}=\cfrac{1\times 3}{2\times 3}=\cfrac{\phantom{3}}{\phantom{6}}$ & 
$\cfrac{3}{6}=\cfrac{3\div 3}{6\div 3}=\cfrac{\phantom{1}}{\phantom{2}}$ \\

\hline
$\cfrac{1}{2}=\cfrac{1\times 4}{2\times 4}=\cfrac{\phantom{4}}{\phantom{8}}$ & 
$\cfrac{4}{8}=\cfrac{4\div \phantom{4}}{8\div \phantom{4}}=\cfrac{1}{2}$ \\

\hline
$\cfrac{1}{2}=\cfrac{1\times \phantom{5}}{2\times \phantom{5}}=\cfrac{5}{10}$ & 
$\cfrac{5}{10}=\cfrac{5\div \phantom{5}}{10\div \phantom{5}}=\cfrac{1}{2}$ \\

\hline

\end{tabular}

\flushleft

จากแนวคิดในการหาเศษส่วนที่เท่ากันข้างต้น กับการเขียนอัตราส่วน
\(1:2,\ 2:4,\ 3:6,\ 4:8\) และ \(5:10\)
โดยการนำจำนวนเดียวกันมาคูณแต่ละจำนวนของอัตราส่วน \(1:2\)
หรือการนำจำนวนเดียวกันมาหารแต่ละจำนวนของอัตราส่วนข้างต้น ทำให้สังเกตได้ว่า
อาจเขียนอัตราส่วนให้อยู่ในรูปเศษส่วนได้
ซึ่งการเขียนอัตราส่วนในรูปเศษส่วนนี้ทำให้สะดวกต่อการนำไปคำนวณต่อ

\centering
\begin{tcolorbox}[colback=white!5!white,
colframe=red!70!black,width=13cm]

อัตราส่วนของปริมาณ $a$ ต่อปริมาณ $b$ นอกจากเขียนแทนด้วย $a:b$ ยังสามารถเขียนในรูปเศษส่วนได้เป็น $\cfrac{a}{b}$

\end{tcolorbox}

\flushleft

จากแนวคิดข้างต้น อัตราส่วนที่ได้จากการคูณจำนวนแรกและจำนวนหลังของอัตราส่วน \(a:b\)
ด้วยจำนวนเดียวกันที่ไม่ใช่ศูนย์ หรือการหารจำนวนแรกและจำนวนหลังของอัตราส่วน \(a:b\)
ด้วยจำนวนเดียวกันที่ไม่ใช่ศูนย์จะเป็น \textbf{\textcolor{NavyBlue}{อัตราส่วนที่เท่ากัน}}
กับอัตราส่วน \(a:b\) ดังนั้น

\centering

\begin{tabular}{llllllllll}

     & $1:2$          & = & $2:4$ & = & $3:6$ & = & $4:8$ & = & $5:6$ \\
หรือ & $\cfrac{1}{2}$ & = & \makebox[1.5cm]{\dotfill}      & = & \makebox[1.5cm]{\dotfill}       & = & \makebox[1.5cm]{\dotfill}       & = &     \makebox[1.5cm]{\dotfill} \\

\end{tabular}

\flushleft

\subsubsection{การหาอัตราส่วนที่เท่ากัน}\label{uxe01uxe32uxe23uxe2buxe32uxe2duxe15uxe23uxe32uxe2auxe27uxe19uxe17uxe40uxe17uxe32uxe01uxe19}

\centering

\begin{tcolorbox}[colback=white!5!white,
colframe=red!70!black, width=13cm]

\textbf{หลักการคูณ}

เมื่อคูณแต่ละจำนวนในอัตราส่วนใดด้วยจำนวนเดียวกันโดยที่จำนวนนั้นไม่เท่ากับศูนย์จะได้อัตราส่วนใหม่ที่เท่ากับอัตราส่วนเดิม


\tcblower

\textbf{หลักการหาร}

เมื่อหารแต่ละจำนวนในอัตราส่วนใดด้วยจำนวนเดียวกันโดยที่จำนวนนั้นไม่เท่ากับศูนย์จะได้อัตราส่วนใหม่ที่เท่ากับอัตราส่วนเดิม


\end{tcolorbox}

\flushleft

\renewcommand{\arraystretch}{1.5}

\begin{longtable*}[l]{lp{13cm}}

\textbf{ตัวอย่างที่ 1} & จงหาอัตราส่วนที่เท่ากับอัตราส่วน $5:9$ มาอีก $5$ อัตราส่วน โดยใช้หลักการคูณ \\
\textbf{วิธีทำ}     &   \dotfill        \\
                  &   \dotfill  \\   
                  &   \dotfill  \\      
                  &   \dotfill  \\      
                  &   \dotfill  \\      
                  &   \dotfill  \\ 
                  &   ดังนั้น อัตราส่วนอีก $5$ อัตราส่วนที่เท่ากับอัตราส่วน $5:9$ ได้แก่ \\  
                  &   \dotfill  \\   

%comment newpage
\newpage

\textbf{ตัวอย่างที่ 2} & จงหาอัตราส่วนที่เท่ากับอัตราส่วน $\cfrac{144}{180}$ มาอีก $3$ อัตราส่วน โดยใช้หลักการหาร \\
\textbf{วิธีทำ}     &   \dotfill        \\
                  &   \dotfill  \\   
                  &   \dotfill  \\      
                  &   \dotfill  \\      
                  &   \dotfill  \\      
                  &   \dotfill  \\ 
                  &   ดังนั้น อัตราส่วนอีก $3$ อัตราส่วนที่เท่ากับอัตราส่วน $\cfrac{144}{180}$ ได้แก่ \\  
                  &   \dotfill  \\  
                              
\end{longtable*}

\centering

\begin{tcolorbox}[colback=white!5!white,
colframe=red!70!black,width=7cm]
\centering \textbf{ลองทำดู : อัตราส่วนที่เท่ากัน}
\end{tcolorbox}

\flushleft

\textbf{จงหาอัตราส่วนที่เท่ากับอัตราส่วนที่กำหนดให้ในแต่ละข้อมา $3$ อัตราส่วน โดยแต่ละข้อต้องใช้ทั้งหลักการคูณและหลักการหาร}

\renewcommand{\arraystretch}{2.2}

\begin{longtable*}[l]{llp{5.5cm}|llp{5.5cm}}

1.  & $\cfrac{1}{2}$   &           & 2.  & $\cfrac{3}{4}$   & \\
    & \textbf{วิธีทำ}  &  \dotfill &     & \textbf{วิธีทำ}  & \dotfill\\
    &               &  \dotfill &     &                 & \dotfill\\ 
    &               &  \dotfill &     &                 & \dotfill\\ 
    &               &  \dotfill &     &                 & \dotfill\\ 
    & \textbf{ตอบ}  &  \dotfill &     & \textbf{ตอบ}  & \dotfill\\        

3.  & $\cfrac{5}{9}$   &           & 4.  & $1:5$   & \\
    & \textbf{วิธีทำ}  &  \dotfill &     & \textbf{วิธีทำ}  & \dotfill\\
    &               &  \dotfill &     &                 & \dotfill\\ 
    &               &  \dotfill &     &                 & \dotfill\\ 
    &               &  \dotfill &     &                 & \dotfill\\ 
    & \textbf{ตอบ}  &  \dotfill &     & \textbf{ตอบ}  & \dotfill\\  
%comment newpage

\newpage

5.  & $8:12$   &           & 6.  & $\cfrac{1.1}{3.3}$   & \\
    & \textbf{วิธีทำ}  &  \dotfill &     & \textbf{วิธีทำ}  & \dotfill\\
    &               &  \dotfill &     &                 & \dotfill\\ 
    &               &  \dotfill &     &                 & \dotfill\\ 
    &               &  \dotfill &     &                 & \dotfill\\ 
    & \textbf{ตอบ}  &  \dotfill &     & \textbf{ตอบ}  & \dotfill\\      

7.  & $\cfrac{0.2}{0.05}$   &           & 8.  & $\cfrac{1.5}{2.5}$   & \\
    & \textbf{วิธีทำ}  &  \dotfill &     & \textbf{วิธีทำ}  & \dotfill\\
    &               &  \dotfill &     &                 & \dotfill\\ 
    &               &  \dotfill &     &                 & \dotfill\\ 
    &               &  \dotfill &     &                 & \dotfill\\ 
    & \textbf{ตอบ}  &  \dotfill &     & \textbf{ตอบ}  & \dotfill\\   

9.  & $\cfrac{72}{144}$   &           & 10.  & $\cfrac{63}{108}$   & \\
    & \textbf{วิธีทำ}  &  \dotfill &     & \textbf{วิธีทำ}  & \dotfill\\
    &               &  \dotfill &     &                 & \dotfill\\ 
    &               &  \dotfill &     &                 & \dotfill\\ 
    &               &  \dotfill &     &                 & \dotfill\\ 
    & \textbf{ตอบ}  &  \dotfill &     & \textbf{ตอบ}  & \dotfill\\  
%comment newpage

\newpage

11.  & $\cfrac{1}{2}:\cfrac{3}{4}$   &           & 12.  & $\cfrac{2}{3}:\cfrac{5}{9}$   & \\
    & \textbf{วิธีทำ}  &  \dotfill &     & \textbf{วิธีทำ}  & \dotfill\\
    &               &  \dotfill &     &                 & \dotfill\\ 
    &               &  \dotfill &     &                 & \dotfill\\ 
    &               &  \dotfill &     &                 & \dotfill\\ 
    & \textbf{ตอบ}  &  \dotfill &     & \textbf{ตอบ}  & \dotfill\\  

\end{longtable*}

\subsubsection{การตรวจสอบการเท่ากันของอัตราส่วนโดยใช้การคูณไขว้}\label{uxe01uxe32uxe23uxe15uxe23uxe27uxe08uxe2auxe2duxe1auxe01uxe32uxe23uxe40uxe17uxe32uxe01uxe19uxe02uxe2duxe07uxe2duxe15uxe23uxe32uxe2auxe27uxe19uxe42uxe14uxe22uxe43uxe0auxe01uxe32uxe23uxe04uxe13uxe44uxe02uxe27}

นักเรียนเคยทราบมาแล้วว่า เราสามารถใช้การคูณไขว้ในการตรวจสอบการเท่ากันของเศษส่วน
ในทำนองเดียวกันเราสามารถใช้การคูณไขว้ในการตรวจสอบการเท่ากันของอัตราส่วนได้

โดยทั่วไปเมื่อ \(a,\ b,\ c\) และ \(d\) เป็นจำนวนบวก
เราสามารถตรวจสอบการเท่ากันของอัตราส่วน \(\cfrac{a}{b}\) กับ \(\cfrac{c}{d}\)
ด้วยการคูณไข้แล้วพิจารณาผลคูณไขว้ \(a\times d\) และ \(b\times c\) ตามหลักการดังนี้

\centering
\begin{tcolorbox}[colback=white!5!white,
colframe=red!70!black,width=8cm]

1. ถ้า $a\times d=b\times c$ แล้ว $\cfrac{a}{b}=\cfrac{c}{d}$

2. ถ้า $a\times d\not= b\times c$ แล้ว $\cfrac{a}{b}\not=\cfrac{c}{d}$

\end{tcolorbox}

\flushleft

จากหลักการข้างต้นทั้ง 2 ข้อ ทำให้ได้ข้อสรุปต่อไปอีกด้วยว่า

\centering
\begin{tcolorbox}[colback=white!5!white,
colframe=red!70!black,width=7cm]

ถ้า $\cfrac{a}{b}=\cfrac{c}{d}$ แล้ว $a\times d=b\times c$

\end{tcolorbox}

\flushleft

\renewcommand{\arraystretch}{1.5}

\begin{longtable*}[l]{lp{13cm}}

\textbf{ตัวอย่างที่ 3} & จงพิจารณาว่าอัตราส่วนที่กำหนดให้แต่ละข้อเป็นอัตราส่วนที่เท่ากันหรือไม่ \\
                  & 1) $\cfrac{2}{6}$ และ $\cfrac{15}{45}$ \\
                  & 2) $\cfrac{3}{7}$ และ $\cfrac{6}{10}$ \\

\textbf{วิธีทำ}     &   \dotfill  \\   
                  &   \dotfill  \\      
                  &   \dotfill  \\      
                  &   \dotfill  \\ 
                  &   \dotfill  \\
                  &   \dotfill  \\  
          
                              
\end{longtable*}

\centering

\begin{tcolorbox}[colback=white!5!white,
colframe=red!70!black,width=8cm]
\centering \textbf{ลองทำดู : การตรวจสอบอัตราส่วนที่เท่ากัน}
\end{tcolorbox}

\flushleft

\textbf{จงพิจารณาอัตราส่วนที่กำหนดให้แต่ละข้อต่อไปนี้ว่าเป็นอัตราส่วนที่เท่ากันหรือไม่ พร้อมทั้งแสดงวิธีคิด}

\renewcommand{\arraystretch}{2.2}

\begin{longtable*}[l]{llp{5.5cm}|llp{5.5cm}}

1.  &   \multicolumn{2}{l}{$2:5$ และ $10:25$}           & 2.  & \multicolumn{2}{l}{$15:24$ และ $5:8$} \\
    & \textbf{วิธีทำ}  &  \dotfill &     & \textbf{วิธีทำ}  & \dotfill\\
    &               &  \dotfill &     &                 & \dotfill\\ 
    &               &  \dotfill &     &                 & \dotfill\\ 
    &               &  \dotfill &     &                 & \dotfill\\ 
    & \textbf{ตอบ}  &  \dotfill &     & \textbf{ตอบ}  & \dotfill\\        

3.  & \multicolumn{2}{l}{$\cfrac{7}{11}$ และ $\cfrac{49}{121}$}           & 4.  & \multicolumn{2}{l}{$\cfrac{24}{56}$ และ $\cfrac{36}{60}$} \\
    & \textbf{วิธีทำ}  &  \dotfill &     & \textbf{วิธีทำ}  & \dotfill\\
    &               &  \dotfill &     &                 & \dotfill\\ 
    &               &  \dotfill &     &                 & \dotfill\\ 
    &               &  \dotfill &     &                 & \dotfill\\ 
    & \textbf{ตอบ}  &  \dotfill &     & \textbf{ตอบ}  & \dotfill\\  


5.  & \multicolumn{2}{l}{$\cfrac{4}{1.5}$ และ $\cfrac{8}{3}$}          & 6.  & \multicolumn{2}{l}{$0.3:5$ และ $6:100$} \\
    & \textbf{วิธีทำ}  &  \dotfill &     & \textbf{วิธีทำ}  & \dotfill\\
    &               &  \dotfill &     &                 & \dotfill\\ 
    &               &  \dotfill &     &                 & \dotfill\\ 
    &               &  \dotfill &     &                 & \dotfill\\ 
    & \textbf{ตอบ}  &  \dotfill &     & \textbf{ตอบ}  & \dotfill\\      

%comment newpage
\newpage

7.  & \multicolumn{2}{l}{$7:2$ และ $24.5:7$}            & 8.  & \multicolumn{2}{l}{$3.5:1.5$ และ $5:2$} \\
    & \textbf{วิธีทำ}  &  \dotfill &     & \textbf{วิธีทำ}  & \dotfill\\
    &               &  \dotfill &     &                 & \dotfill\\ 
    &               &  \dotfill &     &                 & \dotfill\\ 
    &               &  \dotfill &     &                 & \dotfill\\ 
    & \textbf{ตอบ}  &  \dotfill &     & \textbf{ตอบ}  & \dotfill\\   

 

\end{longtable*}




\end{document}
